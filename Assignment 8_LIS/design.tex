\documentclass[UTF8]{ctexart}
\usepackage{geometry}
\geometry{margin=1.5cm, vmargin={0pt,1cm}}
\setlength{\topmargin}{-1cm}
\setlength{\paperheight}{29.7cm}
\setlength{\textheight}{25.3cm}

% useful packages.
\usepackage{amsfonts}
\usepackage{amsmath}
\usepackage{amssymb}
\usepackage{amsthm}
\usepackage{enumerate}
\usepackage{graphicx}
\usepackage{multicol}
\usepackage{fancyhdr}
\usepackage[table]{xcolor}
\usepackage{layout}
\usepackage{float, caption}
\usepackage{xcolor}
\usepackage{listings}
\usepackage{tikz}

% 自定义配色方案,尽量模仿 VS Code 的高亮效果
\definecolor{codegreen}{rgb}{0,0.6,0}
\definecolor{codeblue}{rgb}{0,0,0.9}
\definecolor{codepurple}{rgb}{0.58,0,0.82}
\definecolor{codered}{rgb}{0.8,0,0}
\definecolor{backcolor}{rgb}{0.95,0.95,0.95}

% lstlisting 的风格设置
\lstdefinestyle{vscode}{
	backgroundcolor=\color{backcolor},   % 背景颜色
	commentstyle=\color{codegreen},     % 注释颜色
	keywordstyle=\color{codeblue}\bfseries, % 关键字颜色
	numberstyle=\tiny\color{gray},      % 行号颜色
	stringstyle=\color{codered},        % 字符串颜色
	basicstyle=\ttfamily\footnotesize,  % 基本字体
	breakatwhitespace=false,            % 仅在空格处断行
	breaklines=true,                    % 自动换行
	captionpos=b,                       % 标题位置(bottom)
	keepspaces=true,                    % 保持空格
	numbers=left,                       % 显示行号
	numbersep=5pt,                      % 行号与代码间的间隔
	rulecolor=\color{black},            % 框线颜色
	showspaces=false,                   % 不显示空格符号
	showstringspaces=false,             % 不显示字符串中的空格
	showtabs=false,                     % 不显示制表符
	frame=single,                       % 外框
	tabsize=4,                          % 制表符宽度
	escapeinside={(*@}{@*)},            % 特殊字符转义
	morekeywords={*,...}                % 添加更多自定义关键字
}
\lstset{style=vscode}

% some common command
\newcommand{\dif}{\mathrm{d}}
\newcommand{\avg}[1]{\left\langle #1 \right\rangle}
\newcommand{\difFrac}[2]{\frac{\dif #1}{\dif #2}}
\newcommand{\pdfFrac}[2]{\frac{\partial #1}{\partial #2}}
\newcommand{\OFL}{\mathrm{OFL}}
\newcommand{\UFL}{\mathrm{UFL}}
\newcommand{\fl}{\mathrm{fl}}
\newcommand{\op}{\odot}
\newcommand{\Eabs}{E_{\mathrm{abs}}}
\newcommand{\Erel}{E_{\mathrm{rel}}}

\begin{document}
	
	\pagestyle{fancy}
	\fancyhead{}
	\lhead{高凌溪, 3210105373}
	\chead{2024DS 最长递增子序列问题}
	\rhead{\today}
	\begin{abstract}
		本次作业主要设计了如何寻找一个最长的严格递增的子列。
	\end{abstract}
	\section{设计思路(动态规划)}
	\begin{enumerate}
		\item 首先将数列存到一个数组中\texttt{double array[n]=\{$a_1,\ a_2,\ a_3,\ ...,\ a_n$\}}用$d[i]$表示以数组中第$i$个元素结尾的最长严格递增子列的长度,那么$d[i]=max(d[i],d[j]+1),\quad 1 \le j \le i$ 如果$J<i,\ a[j]<a[i]$(说明\texttt{array[i]}能接在\texttt{array[j]}后面)。之后初始化一个元素全为$1$的数组\texttt{d},这里不妨设下标都是从$1$开始(更符合生活中的习惯)。
		\item 伪代码如下:
		\begin{lstlisting}[language=C++, caption={找到最长递增子列长度}, label={lst:operators}]
			double array[n]={a1, a2,a3,...an}
			int d[n];
			std::fill(d,d+n,1);
			for(int i=2; i<=n; ++i){
				for(int j=1; j<=i; ++j){
					if(array[i]>array[j]){
						d[i]=max(array[i],array[j]+1);
					}
				}
			}
			int length=max(d);
		\end{lstlisting}
		这样要求一条最长递增子列的长度,只需要返回数组\texttt{d}中的最大值即可。
		\item 如果要求出某一个最长递增子列,只要找到\texttt{d[i]}中对应$length,length-1,...,1$的下标$i_{length},...i_1$,然后确保是递增子列即可,具体的伪代码如下:
		\begin{lstlisting}
			int i=length;
			double x=0.0;
			std::queue<double> q;
			for(j=n;j>0; --j){
				if(i==length && d[j]==i){
					q.push(array[j]);
					--i;
					x=array[j]
				}
				else if(i<length && d[j]==i && array[j]<x){
					q.push(array[j]);
					--i;
					x=array[j];
				}
			}
		\end{lstlisting}
		最后让队列的元素出队就得到一个严格最长递增子列。
	\end{enumerate}
\section{算法的复杂度分析}
	显然这个算法的空间复杂度是$O(n)$,由于要找到最长子序列相当于要遍历一遍二维矩阵,因此时间复杂度也是$O(n^2)$
\section{一个简单的例子}
假设现在有一个数组,相应的\texttt{array}和\texttt{d}分别如下,可以看出最长的递增子列长度是6,能找出的一个递增子列是\texttt{\{1,3,4,8,9,10\}}
\[
\begin{array}{|c|c|c|c|c|c|c|c|c|c|c|c|}
	\hline
	\textbf{i}&\textbf{1} & \textbf{2} & \textbf{3} & \textbf{4} & \textbf{5} & \textbf{6} & \textbf{7} & \textbf{8} & \textbf{9} & \textbf{10} \\ \hline
	\textbf{array[i]}&10 & 1 & 5 & 3 & 6 & 4 & 8 & 9 & 7 & 10 \\ \hline
	\textbf{d[i]}& 1 & 1 & 2 & 2 & 3 & 3 & 4 & 5 & 4 & 6 \\ \hline
\end{array}
\]

\end{document}
