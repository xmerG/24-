\documentclass[UTF8]{ctexart}
\usepackage{geometry, CJKutf8}
\geometry{margin=1.5cm, vmargin={0pt,1cm}}
\setlength{\topmargin}{-1cm}
\setlength{\paperheight}{29.7cm}
\setlength{\textheight}{25.3cm}

% useful packages.
\usepackage{amsfonts}
\usepackage{amsmath}
\usepackage{amssymb}
\usepackage{amsthm}
\usepackage{enumerate}
\usepackage{graphicx}
\usepackage{multicol}
\usepackage{fancyhdr}
\usepackage{layout}
\usepackage{listings}
\usepackage{float, caption}
\usepackage{xcolor}


\lstset{
	basicstyle=\ttfamily,
	basewidth=0.45em,
	backgroundcolor=\color{lightgray},
	frame=none,
	breaklines=true  % 代码自动换行
}
% some common command
\newcommand{\dif}{\mathrm{d}}
\newcommand{\avg}[1]{\left\langle #1 \right\rangle}
\newcommand{\difFrac}[2]{\frac{\dif #1}{\dif #2}}
\newcommand{\pdfFrac}[2]{\frac{\partial #1}{\partial #2}}
\newcommand{\OFL}{\mathrm{OFL}}
\newcommand{\UFL}{\mathrm{UFL}}
\newcommand{\fl}{\mathrm{fl}}
\newcommand{\op}{\odot}
\newcommand{\Eabs}{E_{\mathrm{abs}}}
\newcommand{\Erel}{E_{\mathrm{rel}}}

\begin{document}

\pagestyle{fancy}
\fancyhead{}
\lhead{高凌溪, 3230100001}
\chead{数据结构与算法第四次作业}
\rhead{\today}

\section{测试程序的设计思路}
\begin{enumerate}
	\item 首先在\lstinline|List.h|文件中我补充了\lstinline|find(const Object &val)|函数和\lstinline|resize()|函数,其中\lstinline|find()|函数输入要查找的值,返回第一个符合条件的值所在的位置,若val不在链表里返回-1。\lstinline|resize()|函数第一个参数输入resize后链表的长度n;如果$n<theSize$,将链表多余的元素直接去掉;如果$n>theSize$,将链表多余部分用第二个参数给出的元素填充。我补充了\lstinline|print()|函数打印列表,便于观察结果。最后为了清楚的看到析构函数的调用,我修改了析构函数并且希望每次析构会输出\lstinline|the process is constructing|.
	\item 为了测试构造函数,我先构造了空链表\lstinline|l1|,测试默认构造函数,打印\lstinline|l1|并利用\lstinline|push_front()|函数向其中依次添加元素0到4,我们希望此时\lstinline|l1|应该形如\lstinline|4 3 2 1 0|,之后测试右值引用的\lstinline|push_front()|函数,利用\lstinline|push_front(10+0)|向\lstinline|l1|添加元素,之后打印\lstinline|l1|观察结果。为了测试动态的\lstinline|front()|函数,我利用\lstinline|l1.front()=1|,将\lstinline|l1|的第一个元素改为了1,然后用\lstinline|l1.pop_back()|函数删除了\lstinline|l1|的最后一个元素,再打印\lstinline|l1|观察与之前的异同。
	\item 为了测试拷贝构造函数,构造了\lstinline|List<int>l6=l1|,利用\lstinline|push_back()|函数向其中添加了元素0到4,这时候我希望\lstinline|l4|应该形如\lstinline|0 1 2 3 4|,为了测试右值引用的\lstinline|push_back|函数,我通过\lstinline|l6.push_back(0+10), l6.push_back(1+10)|添加了10和11两个元素,打印观察。之后为了测试动态的\lstinline|back()|函数,我用\lstinline|l6.back()=0|,将\lstinline|l6|的最后一个元素修改为0,之后测试\lstinline|pop_front()|函数删除\lstinline|l6|的第一个元素,打印\lstinline|l6|观察结果。
	\item  我创建了\lstinline|List<double>l2={1.2,1.3, 1.4, 1.5}|测试列表的初始化功能,又利用\lstinline|const List<double>l3=l2|构造了\lstinline|l2|用来测试静态的\lstinline|front() back()|函数,打印结果。
	\item  为了测试移动构造函数,我构造了\lstinline|List<int>l|,并向其中添加了元素\lstinline|0 1 2|,之后利用\lstinline|List<int>l5(move(l))|将结果移动给\lstinline|l5|,之后我们输出\lstinline|l|观察是否发生了复制,结果显示\lstinline|l|变成了空链表,没有复制,之后利用\lstinline|l5|测试\lstinline|empty(), size()|函数,然后打印\lstinline|l5|,利用\lstinline|l5.resize(6,1)|测试\lstinline|resize()|函数,再利用\lstinline|l5.find(10), l5.find(1)|测试\lstinline|find()|函数。打印。再新建空链表\lstinline|l7|,利用\lstinline|l7=move(l5)|测试移动赋值操作。打印\lstinline|l7 l5|.
	\item 利用两个\lstinline|iterator itr1 itr2|测试左值引用和右值引用的\lstinline|insert() erase()|函数。
	\item 最后我构造了空链表\lstinline|l8|测试\lstinline|begin() end()|函数并向其中添加了一个元素,是为了测试不同情况下的\lstinline|begin() end()|函数。
\end{enumerate}

\section{测试的结果}

测试结果一切正常。

我用 valgrind 进行测试,发现没有发生内存泄露。
最终在终端输出的结果如下:
\begin{lstlisting}
	the list is empty
	11 10 4 3 2 1 0 
	1 10 4 3 2 1 
	0 1 2 3 4 10 11 
	1 2 3 4 10 0 
	the first element in l3 is 1.2
	the last elemnet in l3 is 1.5
	1.2 1.3 1.4 1.5 
	the process is constructing
	1 2 3 
	the list is empty
	show that if l5 is empty 0
	the size of l5 is 3
	0 1 2 
	0 1 2 1 1 1 
	to check if 10 is in l5-1
	to check if 1 is in l51
	the process is constructing
	0 1 2 1 1 1 
	the list is empty
	1.2 0.03 1.3 0.04 1.4 
	1.2 0.03 1.3 0 0.04 1.4 
	the list is empty
	the list is empty
	Single element: 10
	the process is constructing
	the process is constructing
	the process is constructing
	the process is constructing
	the process is constructing
	the process is constructing
	the process is constructing
	the process is constructing
	the process is constructing
	==5563== 
	==5563== HEAP SUMMARY:
	==5563==     in use at exit: 0 bytes in 0 blocks
	==5563==   total heap usage: 63 allocs, 63 frees, 75,192 bytes allocated
	==5563== 
	==5563== All heap blocks were freed -- no leaks are possible
	==5563== 
	==5563== For lists of detected and suppressed errors, rerun with: -s
	==5563== ERROR SUMMARY: 0 errors from 0 contexts (suppressed: 0 from 0)
\end{lstlisting}

\end{document}

%%% Local Variables: 
%%% mode: latex
%%% TeX-master: t
%%% End: 
