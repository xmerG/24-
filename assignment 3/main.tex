\documentclass[a4paper]{article}
\usepackage[affil-it]{authblk}
\usepackage[backend=bibtex,style=numeric]{biblatex}
\usepackage{ctex}
\usepackage{listings}
\usepackage{geometry}
\usepackage{xcolor}
\usepackage{float} 
\usepackage{amsmath}
\usepackage{amssymb}
\usepackage{mathtools}
\geometry{margin=1.5cm, vmargin={0pt,1cm}}
\setlength{\topmargin}{-1cm}
\setlength{\paperheight}{29.7cm}
\setlength{\textheight}{25.3cm}


\begin{document}

\title{数据结构与算法第三次作业报告}


\author{Lingxi Gao
  \thanks{Electronic address: \texttt{3210105373@zju.edu.cn}}}
\affil{Zhejiang University }


\date{\today}

\maketitle






% ============================================
\section*{问题描述}
在目前的设计中,remove()函数会删除currentPos之后的元素,而如果用户想先查看某个元素x是否在单链表中,如果在,则删除该元素,这一点目前的remove()函数无法做到。
\section*{改进思路}
我的思路是重载一个remove(T \_val)函数,实现以下功能:\\
输入\_val的值,如果在链表中,则删除该值;如果不在链表中,则报错"the value you put is not in current list"并终止进程。\\
具体地,首先调用find()函数确认\_val是否在当前SingleLinkedList中,由于执行完find()操作之后,如果\_val在SingleLinkeddList之中,currentpos会指向\_val所在的节点。之后再实现删除currentpos指向的节点即可。详细的代码如下:\\
\begin{lstlisting}[language=C++, caption={改进后的代码}]
	template<typename T>
	void SingleLinkedList<T>::remove(T _val){
		if(find(_val)){
			Node* p=head;
			if(head!=currentPos){
				while (p->next!=currentPos){
					p=p->next;
				}
				Node* p1=p->next->next;
				delete p->next;
				p->next=p1;
				currentPos=p;     
				--size;
			}
			else{
				Node* p1=head->next;
				delete currentPos;
				head=p1;
				currentPos=head;
			}
		}
		else{
			cerr<<"the value you put is not in current single linked list"<<endl;
			exit(1);
		}
	}
\end{lstlisting}
\begin{lstlisting}[language=C++, caption={测试}]
	#include "LinkedList.h"
	int main()
	{
		SingleLinkedList<int> a1{1,2,3,4,5};
		a1.remove(3);
	    a1.remove(1);
		a1.printList();
		return 0;
	};
\end{lstlisting}
\section*{运行结果}
最终运行结果如下:


2	4	5




\end{document}