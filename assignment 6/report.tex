\documentclass[UTF8]{ctexart}
\usepackage{geometry, CJKutf8}
\geometry{margin=1.5cm, vmargin={0pt,1cm}}
\setlength{\topmargin}{-1cm}
\setlength{\paperheight}{29.7cm}
\setlength{\textheight}{25.3cm}

% useful packages.
\usepackage{amsfonts}
\usepackage{amsmath}
\usepackage{amssymb}
\usepackage{amsthm}
\usepackage{enumerate}
\usepackage{graphicx}
\usepackage{multicol}
\usepackage{fancyhdr}
\usepackage{layout}
\usepackage{listings}
\usepackage{float, caption}
\usepackage{xcolor}

\lstdefinestyle{mystyle}{
	backgroundcolor=\color{white},   % 背景颜色
	commentstyle=\color{green},       % 注释颜色
	keywordstyle=\color{magenta},     % 关键字颜色
	numberstyle=\tiny\color{gray},    % 行号样式
	stringstyle=\color{red},          % 字符串颜色
	basicstyle=\ttfamily\footnotesize, % 基本字体
	breakatwhitespace=false,           % 断行时不在空格处断行
	breaklines=true,                   % 自动换行
	numbers=left,                     % 行号位置
	numbersep=5pt,                   % 行号与代码间距
	frame=single,                     % 单线框
	rulecolor=\color{black},          % 框的颜色
	captionpos=b,                     % 标题位置
	escapeinside={(*@}{@*)}           % 特殊字符转义
}
\lstset{style=mystyle}
% some common command
\newcommand{\dif}{\mathrm{d}}
\newcommand{\avg}[1]{\left\langle #1 \right\rangle}
\newcommand{\difFrac}[2]{\frac{\dif #1}{\dif #2}}
\newcommand{\pdfFrac}[2]{\frac{\partial #1}{\partial #2}}
\newcommand{\OFL}{\mathrm{OFL}}
\newcommand{\UFL}{\mathrm{UFL}}
\newcommand{\fl}{\mathrm{fl}}
\newcommand{\op}{\odot}
\newcommand{\Eabs}{E_{\mathrm{abs}}}
\newcommand{\Erel}{E_{\mathrm{rel}}}

\begin{document}

\pagestyle{fancy}
\fancyhead{}
\lhead{高凌溪, 3210105373}
\chead{2024DS编程作业报告}
\rhead{\today}
\begin{abstract}
	本次编程作业主要重写了\texttt{BST.h}中的\texttt{remove()}函数,实现了按AVL树的方式删除。
\end{abstract}
\section{\texttt{remove()}函数的设计思路}
\begin{enumerate}
	\item
	整体思路:AVL树和BST只有一点不同,即AVL树需要两侧子树的平衡。因此在实现\texttt{remove()}的时候,新增一个\texttt{balance()}函数,用于平衡子树。具体地,用之前BST的\texttt{remove()}方式找到要删除的节点,删掉;然后依次回溯父节点,检查节点是否平衡,不平衡进行\texttt{balance()},知道回溯到root节点停止。由于需要回溯父节点,这次作业不会继续使用上次\texttt{remove()}的方式,改用递归。
	\item 
	\texttt{balance()}的具体实现:如果是左子树的左节点引起失衡,右旋一次;右子树的右孩子引起失衡,左旋一次;左子树的右节点引起失衡,左旋一次,右旋一次;右子树的左节点引起失衡,右旋一次,左旋一次。其中左旋和右旋分别通过\texttt{singleLeftRotate}和\texttt{singleRightRotate}实现。
\end{enumerate}





\end{document}

%%% Local Variables: 
%%% mode: latex
%%% TeX-master: t
%%% End: 
