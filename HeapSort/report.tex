\documentclass[UTF8]{ctexart}
\usepackage{geometry, CJKutf8}
\geometry{margin=1.5cm, vmargin={0pt,1cm}}
\setlength{\topmargin}{-1cm}
\setlength{\paperheight}{29.7cm}
\setlength{\textheight}{25.3cm}

% useful packages.
\usepackage{amsfonts}
\usepackage{amsmath}
\usepackage{amssymb}
\usepackage{amsthm}
\usepackage{enumerate}
\usepackage{graphicx}
\usepackage{multicol}
\usepackage{fancyhdr}
\usepackage{layout}
\usepackage{listings}
\usepackage{float, caption}
\usepackage{xcolor}

\lstdefinestyle{mystyle}{
	backgroundcolor=\color{white},   % 背景颜色
	commentstyle=\color{green},       % 注释颜色
	keywordstyle=\color{magenta},     % 关键字颜色
	numberstyle=\tiny\color{gray},    % 行号样式
	stringstyle=\color{red},          % 字符串颜色
	basicstyle=\ttfamily\footnotesize, % 基本字体
	breakatwhitespace=false,           % 断行时不在空格处断行
	breaklines=true,                   % 自动换行
	numbers=left,                     % 行号位置
	numbersep=5pt,                   % 行号与代码间距
	frame=single,                     % 单线框
	rulecolor=\color{black},          % 框的颜色
	captionpos=b,                     % 标题位置
	escapeinside={(*@}{@*)}           % 特殊字符转义
}
\lstset{style=mystyle}
% some common command
\newcommand{\dif}{\mathrm{d}}
\newcommand{\avg}[1]{\left\langle #1 \right\rangle}
\newcommand{\difFrac}[2]{\frac{\dif #1}{\dif #2}}
\newcommand{\pdfFrac}[2]{\frac{\partial #1}{\partial #2}}
\newcommand{\OFL}{\mathrm{OFL}}
\newcommand{\UFL}{\mathrm{UFL}}
\newcommand{\fl}{\mathrm{fl}}
\newcommand{\op}{\odot}
\newcommand{\Eabs}{E_{\mathrm{abs}}}
\newcommand{\Erel}{E_{\mathrm{rel}}}

\begin{document}

\pagestyle{fancy}
\fancyhead{}
\lhead{高凌溪, 3210105373}
\chead{2024DS编程作业报告}
\rhead{\today}
\begin{abstract}
	本次编程作业主要实现了堆排序。
\end{abstract}
\section{设计思路}
\begin{enumerate}
	\item
	\texttt{HeapSort}设计思路:\\
	定义了\texttt{adjust(vector<T> \&v, int i, const int \&n)}函数和\texttt{sort\_Heap()}函数。\texttt{adjust()}将\texttt{vector v}以$i$ 为根节点,长度为$n$的部分调整为最大堆。具体实现是通过小元素的下沉实现的。\texttt{sort\_Heap()}函数首先将\texttt{vector}变成最大堆:具体的,从最后一个有子节点的元素开始,向上一次调整。然后将最大的元素与最后一个元素交换位置,利用堆的性质,对第一个元素执行\texttt{adjust()}函数。
	\item 
	测试文件设计思路:\\
	为了保证测试结果有一定的可重复性,设置随机数种子。分别生成了随机序列、按照升序排列的序列、按降序排列的序列和有部分重复的序列\texttt{v}。然后用复制的方式生成一个一样的子列\texttt{v1},\texttt{v}用\texttt{std::sort\_heap()}排序,\texttt{v1}用我写的\texttt{sort\_Heap()}函数排序。利用\texttt{void Checkorder(const vector<T> \&v)}这个函数来检查是否成功按照升序排列。利用\texttt{Check}类里的\texttt{compareTime()}比较运行时间。
\end{enumerate}
\section{程序运行结果}
\begin{center}
\begin{tabular}{|c|c|c|}
	\hline & my sort\_Heap time & std::sort\_heap time \\
	\hline random sequence & 313 ms & 502 ms \\
	\hline ordered sequence & 233 ms & 420 ms \\
	\hline reverse sequence & 210 ms & 429 ms \\
	\hline repetitive sequence & 275 ms & 465 ms \\
	\hline
\end{tabular}
\end{center}
\section{结果分析}
可以看出我写的\texttt{sort\_Heap}函数要比\texttt{std::sort\_heap}函数速度快,我猜测原因如下:
\begin{enumerate}
	\item 
	查询C++库的实现代码可以看到
	\begin{lstlisting}
		template <class _RandomAccessIterator, class _Compare>
		inline _LIBCPP_HIDE_FROM_ABI _LIBCPP_CONSTEXPR_SINCE_CXX20 void
		sort_heap(_RandomAccessIterator __first, _RandomAccessIterator __last, _Compare __comp)
	\end{lstlisting}
	\texttt{std::sort\_heap()}函数不仅可以对\texttt{vector}的容器进行排序,也可以对其他容器里的数据排序,而我实现的\texttt{sort\_Heap()}只针对\texttt{vector},因此可以充分的利用\texttt{vector}的特点
\end{enumerate}

\end{document}

%%% Local Variables: 
%%% mode: latex
%%% TeX-master: t
%%% End: 
