\documentclass[UTF8]{ctexart}
\usepackage{geometry, CJKutf8}
\geometry{margin=1.5cm, vmargin={0pt,1cm}}
\setlength{\topmargin}{-1cm}
\setlength{\paperheight}{29.7cm}
\setlength{\textheight}{25.3cm}

% useful packages.
\usepackage{amsfonts}
\usepackage{amsmath}
\usepackage{amssymb}
\usepackage{amsthm}
\usepackage{enumerate}
\usepackage{graphicx}
\usepackage{multicol}
\usepackage{fancyhdr}
\usepackage{layout}
\usepackage{listings}
\usepackage{float, caption}
\usepackage{xcolor}

\lstdefinestyle{mystyle}{
	backgroundcolor=\color{white},   % 背景颜色
	commentstyle=\color{green},       % 注释颜色
	keywordstyle=\color{magenta},     % 关键字颜色
	numberstyle=\tiny\color{gray},    % 行号样式
	stringstyle=\color{red},          % 字符串颜色
	basicstyle=\ttfamily\footnotesize, % 基本字体
	breakatwhitespace=false,           % 断行时不在空格处断行
	breaklines=true,                   % 自动换行
	numbers=left,                     % 行号位置
	numbersep=5pt,                   % 行号与代码间距
	frame=single,                     % 单线框
	rulecolor=\color{black},          % 框的颜色
	captionpos=b,                     % 标题位置
	escapeinside={(*@}{@*)}           % 特殊字符转义
}
\lstset{style=mystyle}
% some common command
\newcommand{\dif}{\mathrm{d}}
\newcommand{\avg}[1]{\left\langle #1 \right\rangle}
\newcommand{\difFrac}[2]{\frac{\dif #1}{\dif #2}}
\newcommand{\pdfFrac}[2]{\frac{\partial #1}{\partial #2}}
\newcommand{\OFL}{\mathrm{OFL}}
\newcommand{\UFL}{\mathrm{UFL}}
\newcommand{\fl}{\mathrm{fl}}
\newcommand{\op}{\odot}
\newcommand{\Eabs}{E_{\mathrm{abs}}}
\newcommand{\Erel}{E_{\mathrm{rel}}}

\begin{document}

\pagestyle{fancy}
\fancyhead{}
\lhead{高凌溪, 3210105373}
\chead{2024DS编程作业报告}
\rhead{\today}
\begin{abstract}
	本次编程作业主要重写了\texttt{BST.h}中的\texttt{remove}函数,实现了二叉搜索树里元素的删除,避免了使用递归的方式。
\end{abstract}
\section{\texttt{BST.h}的设计思路}
\begin{enumerate}
	\item
	首先新增了函数\texttt{BinaryNode *detachMin(BinaryNode *\&t)},主要作用是为了找到以\texttt{t}为root的二叉搜索树中最小元素的节点,并返回该节点。由于最小元素一定在tree的最左边,只要检测\texttt{if(current->left!=nullptr)}即可。同时对节点\lstinline|t|做出以下修改:
	\begin{lstlisting}[language=C++]
		//如果最小元素的节点不是根节点t
		if(parent!=nullptr){
			parent->left=current->right;
		}
		//如果是根节点
		else{
			t=current->right;
		}
	\end{lstlisting}
	这里的\texttt{parent \ current}分别指的是最小元素所在节点的父节点和最小元素所在的节点。
	\item 
	修改\text{remove}函数。这里分为五种情况:
	\begin{enumerate}
		\item 最简单的情况:二叉树为空,直接\texttt{return}
		\item 若二叉树不为空,首先要找到需要删除的元素所在的节点,因此要追踪当前节点和父节点。
		\begin{lstlisting}[language=C++]
		BinaryNode *parent=nullptr;
		BinaryNode *current=t;
		\end{lstlisting}
		若要删除的元素不在tree里面,直接\texttt{return}
		\item 如果要删除的元素只有一个子节点,这时候直接删除该元素,用该元素的子节点取代当前节点。
		\item 如果要删除的元素有两个子节点,而且要删除的元素就是根节点:此时根节点由右子节点中最小元素所在的节点取代,根节点的左子节点不变,右子节点的结构由\texttt{BinaryNode *detachMin(current->right)}之后的结果给出。
		\begin{lstlisting}[language=C++]
		if(current->left!=nullptr && current->right!=nullptr){
		BinaryNode *newroot=detachMin(current->right);
		newroot->right=current->right;
		//此时要删除的节点就是root,current指向root
		if(parent==nullptr){
			t=newroot;
		}
		\end{lstlisting}
		\item 若要删除的元素有两个子节点,且不是根节点,这时候与(c)相比需要额外将新的根节点接到原来的父节点上。
		\begin{lstlisting}[language=C++]
		//要删除的节点还有父节点,这时候要把newroot接到父节点上
		else if(x<parent->element){
			parent->left=newroot;
		}
		else{
			parent->right=newroot;
		}
		\end{lstlisting}
	\end{enumerate}
\end{enumerate}

\section{\texttt{test\_BST.cpp}测试思路}
\begin{enumerate}
	\item 
	首先插入\texttt{10 5 15 3 7 12 18 20 1 4}到空的二叉搜索树中,打印检测是否插入成功。之后删除元素\texttt{5}测试情况(e),删除元素\texttt{7}测试情况(c),删除元素\texttt{15}测试情况(e),删除元素\texttt{10}测试情况(d)。每次都打印观察结果。
\end{enumerate}

\section{测试的结果}

测试结果一切正常。

最终在终端输出的结果如下:
\begin{lstlisting}
g++ -o test test_BST.cpp
Running test:
./test
Initial Tree:
1
3
4
5
7
10
12
15
18
20
Minimum element: 1
Maximum element: 20
Contains 7? Yes
Contains 20? Yes
Tree after removing 5:
1
3
4
7
10
12
15
18
20
Tree after removing 7:
1
3
4
10
12
15
18
20
Tree after removing 15:
1
3
4
10
12
18
20
Tree after removing 10:
1
3
4
12
18
20
\end{lstlisting}

\end{document}

%%% Local Variables: 
%%% mode: latex
%%% TeX-master: t
%%% End: 
